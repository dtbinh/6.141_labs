\documentclass[letterpaper,12pt]{article}
\usepackage{graphicx}
\usepackage{amsmath}
\usepackage{subfig}
\usepackage{fullpage}
\usepackage{setspace}

\title{Team 6 Challenge Proposal}
\author{6.141\\
	Spring 2012\\ \\
	Neil Forrester\\
	Dan Gonzalez\\
	Raghavendra Srinivasan\\
	James Wiken}

\onehalfspacing				

\begin{document}
\begin{singlespacing}
\maketitle
\thispagestyle{empty}
\newpage

\tableofcontents
\listoffigures
\listoftables
\thispagestyle{empty}
\newpage

\end{singlespacing}

\setcounter{page}{1}

\section{Problem Statement}
The scenario we wish to emulate can be pictured as a robot being parachuted onto the surface of Mars with the purpose of building a shelter of some form.  Building materials have also been parachuted down and are distributed across the Martian landscape.  The robot has been provided with a map of the local area and the locations of some of the building materials.  However, some of the materials have not landed in the designated areas and thus do not show up on the map despite being useful, or perhaps essential, to the construction of the shelter.  The robot must navigate the countryside, gather materials, and use said materials to build a structure.

Similarly, our robot must perform four main tasks.  

\begin{enumerate}
 \item {\bf Navigate about an area}: The robot will be placed in a maze of some undefined size and complexity.  The robot will be provided a map of the maze beforehand and must be able to autonomously navigate the area.
 \item {\bf Locate and collect building materials}: Building materials in the form of differently colored blocks will be placed around the area.  Locations of some of the blocks will be provided with the map, but some subset of the blocks are not marked on the map.  The robot must be able to locate, recognize, and collect some subset of the blocks.
 \item {\bf Transport building materials to construction zone}: The robot must be able to move the collected blocks from their initial positions to an autonomously determined construction point.
 \item {\bf Assemble materials into a structure}: Using the collected building materials, the robot must build a structure.  The structure is roughly defined as a collection of blocks within the construction zone.  The structure can range from simple, such as a pile or a low wall, to complex, such as an enclosed room or an archway.  
\end{enumerate}

The integration of these four different abilities into a single task makes this project both interesting and challenging.  The elements needed to solve this problem can be used to further other robot applications like those listed above.  Conversely, other applications can provide at least partial solutions to this problem. 

Before proceeding with the design of a solution to the given problem statement, a series of assumptions must be made.  These assumptions can be divided into three main categories: assumptions about the environment, assumptions about the building blocks, and assumptions about the robot.

\subsection{Assumptions about the environment}

\subsection{Assumptions about the building blocks}

\subsection{Assumptions about the robot}


\section{High level approach and mechanical structure}
\section{Systems}
\subsection{most}
\subsection{of}
\subsection{the}
\subsection{robot's}
\subsection{systems}
\section{Schedule}
\section{Conclusion}

\end{document}
